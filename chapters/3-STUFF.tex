%% 个人主要研究内容

\chapter{研究内容}
	研究内容。
	
	本篇论文的核心研究内容。主要包括理论分析、计算方法等。

\section{插图示例}
    {\bfseries 单张图片示例,如图\ref{fig:10GHz}。}
    
    \pic{10~GHz~倍频链结构图}{0.8\textwidth}{10GHz}{\label{fig:10GHz}}
    
    {\bfseries 两张分图,如图\ref{fig:twopics}。该图包含左右两幅分图,分别为函数图象一,如图\ref{fig:fisher1}所示;和函数图象二,如图\ref{fig:fisher2}。}
    
    \begin{pics}{分图效果}{\label{fig:twopics}}
        \subpic{函数图象一}{0.3\textwidth}{fisher1}{\label{fig:fisher1}}
        \subpic{函数图象二}{0.3\textwidth}{fisher2}{\label{fig:fisher2}}
    \end{pics}
    
    {\bfseries 一组相关图片(三张分图),如图\ref{fig:3pics}。}校训是学校办学理念、育人理念、内在精神的高度浓缩,是学校的精神灵魂和象征。五十载厚德载物,成电积淀出了深厚的人文情怀和科学精神,形成了厚重的办学理念和鲜明的办学特色。许多年来,成电人一直在思考、挖掘、梳理、总结和凝练这笔宝贵的精神财富,并希望形成集中反映成电人精神追求的、体现学校办学理念和特色的校训。
    
    \begin{pics}{综合流程图}{\label{fig:3pics}}
        \subpic{流程一}{0.25\textwidth}{flow5-1}{\label{fig:flow1}}
        \subpic{流程二}{0.25\textwidth}{flow5-2}{\label{fig:flow2}}
        \subpic{流程三}{0.25\textwidth}{flow5-3}{\label{fig:flow3}}
    \end{pics}
    
    {\bfseries 两张并列图片,如图\ref{fig:apk}和\ref{fig:tulips}。}自~02~年到~05~年以来,学校先后两次开展了校训征集活动,广大师生积极参与,提出了许多好的建议。同时在建校五十周年前夕,学校还组织开展了“成电精神大家谈活动”,通过网谈、笔谈、访谈、座谈等多种方式探讨和凝练成电精神。
    
    \begin{mpics}
        \minipic{apk~项目结构}{0.9\textwidth}{apk}{\label{fig:apk}}
        \minipic{郁金香}{0.9\textwidth}{tulips}{\label{fig:tulips}}
    \end{mpics}
    
    通过这些活动,引发了广大师生对大学精神、办学理念的积极关注和深入思考,大家对学校办学理念和育人理念的认识逐步深化,对成电精神的认识逐渐趋同。时任党委书记胡树祥在电子科大报第~707~期《弘扬成电精神》的文章里初步归纳和阐述了“爱国奉献、艰苦奋斗,求真务实、知行合一,追求卓越、争创一流”的成电精神,时任校长邹寿彬在庆祝电子科技大学建校五十周年大会上的讲话中重申了成电精神,并阐述了“大楼、大师、大气、大为”的办学理念,这些充分体现了成电人建设高水平研究性大学的共同思想基础和价值追求,在广大师生中产生了强烈的共鸣,形成了高度共识。

\section{表格示例}
    
    {\bfseries 页内三线表示例,如表\ref{tbl:yzhc}所示。}在形成广泛共识的基础上,根据校训的形式特点,我们进一步进行了文字的归纳和提炼,将校训概括为“求实求真,大气大为”。其基本涵义简而言之就是要有实事求是的科学态度,脚踏实地的工作作风,追求真理的科学精神,海纳百川的胸襟气度,志存高远的远大志向,追求卓越的精神境界。
    
    \utable[htbp]{0.8\textwidth}{ccccc}{研制合成器的实测频谱指标}{
        频偏 & \shortstack{11.1GHz\\(dBc/Hz)} & \shortstack{12.1GHz\\(dBc/Hz)} 
            & \shortstack{13.1GHz\\(dBc/Hz)} & \shortstack{设计指标\\(dBc/Hz)}\\
    }{
        100Hz  &-76    &-74    &-77   &-65  \\
        1kHz   &-84    &-83    &-84.6 &-83  \\
        10kHz  &-89.3  &-89    &-88.6 &-88  \\
        100kHz &-100   &-100   &-98.6 &-98  \\
        杂散    &-66    &-64.5  &-64   &-60  \\
    }{}{\label{tbl:yzhc}}
    
	{\bfseries 通过外部软件制作的表格,转换为图片后可作为图片表格插入,如表\ref{tbl:excel}所示。}“求实”是大学的内在品格。它是一所大学在办学实践中积淀形成的作风和品格。所谓“求实”就是在科学和真理面前要有实事求是的态度,在工作上要有实干的精神和实在的结果,在做人上要有敦实厚重的品格。

    \pictable{用~Excel~制作的表格}{0.8\textwidth}{excel}{\label{tbl:excel}}
    
    {\bfseries 有时需要引用大段数据或参量对文章内容作出说明,用到跨页的表格。跨页三线表示例,如表\ref{tbl:long}所示。}这种求实风格和气质的形成,首先得益于学校一贯倡导的“团结、勤奋、求实、创新”的优良作风,使成电形成了务实的工作作风和实事求是的科学态度,即使在十年“文革”期间,也取得了承担科研课题~237~项、获得成果~60~余项、发表论文~500~余篇这样在全国高校中较为突出的成就;再如~1970~年开始立时四年的全国彩色电视大会战,正是成电人脚踏实地、奋勇拼搏的一个缩影。其次,一大批德高望重的学者和专家,如前苏联专家列别捷夫、罗金斯基以及我校林为干、谢立惠、顾德仁等一批学术前辈,他们崇高的人格魅力、严谨求实的治学态度、潜心研究的科学精神、一丝不苟的工作作风深刻影响着一代代成电人。再次,成电学子走向社会,在母校求实精神的滋养下,养成了勤勉踏实的工作作风,赢得了社会对成电校友“为人厚实,知识扎实,做事踏实”的赞誉,从另一个侧面生动地折射出成电人的求实之风。

    \longutable{跨页长表格}{3}{
        项目  &频率  &技术\\
    }{
        802.11b\tnote{a}    &2.4~GHz     &DSSS\\
        蓝牙                  &2.4~GHz     &FHSS\\
        HomeRF              &2.4~GHz     &FHSS\\
        802.11b\tnote{a}    &2.4~GHz     &DSSS\\
        蓝牙                  &2.4~GHz     &FHSS\\
        HomeRF              &2.4~GHz     &FHSS\\
        802.11b\tnote{a}    &2.4~GHz     &DSSS\\
        蓝牙                  &2.4~GHz     &FHSS\\
        HomeRF              &2.4~GHz     &FHSS\\
        802.11b\tnote{a}    &2.4~GHz     &DSSS\\
        蓝牙                  &2.4~GHz     &FHSS\\
        HomeRF              &2.4~GHz     &FHSS\\
        802.11b\tnote{a}    &2.4~GHz     &DSSS\\
        蓝牙                  &2.4~GHz     &FHSS\\
        HomeRF              &2.4~GHz     &FHSS\\
        802.11b\tnote{a}    &2.4~GHz     &DSSS\\
        蓝牙                  &2.4~GHz     &FHSS\\
        HomeRF              &2.4~GHz     &FHSS\\
        802.11b\tnote{a}    &2.4~GHz     &DSSS\\
        蓝牙                  &2.4~GHz     &FHSS\\
        HomeRF              &2.4~GHz     &FHSS\\
        802.11b\tnote{a}    &2.4~GHz     &DSSS\\
        蓝牙                  &2.4~GHz     &FHSS\\
        HomeRF              &2.4~GHz     &FHSS\\
        802.11b\tnote{a}    &2.4~GHz     &DSSS\\
        蓝牙                  &2.4~GHz     &FHSS\\
        HomeRF              &2.4~GHz     &FHSS\\
        802.11b\tnote{a}    &2.4~GHz     &DSSS\\
        蓝牙                  &2.4~GHz     &FHSS\\
        HomeRF              &2.4~GHz     &FHSS\\
        802.11b\tnote{a}    &2.4~GHz     &DSSS\\
        蓝牙                  &2.4~GHz     &FHSS\\
        HomeRF              &2.4~GHz     &FHSS\\
        802.11b\tnote{a}    &2.4~GHz     &DSSS\\
        蓝牙                  &2.4~GHz     &FHSS\\
        HomeRF              &2.4~GHz     &FHSS\\
        802.11b\tnote{a}    &2.4~GHz     &DSSS\\
        蓝牙                  &2.4~GHz     &FHSS\\
        HomeRF              &2.4~GHz     &FHSS\\
        802.11b\tnote{a}    &2.4~GHz     &DSSS\\
        蓝牙                  &2.4~GHz     &FHSS\\
        HomeRF              &2.4~GHz     &FHSS\\
        802.11b\tnote{a}    &2.4~GHz     &DSSS\\
        蓝牙                  &2.4~GHz     &FHSS\\
        HomeRF              &2.4~GHz     &FHSS\\
    }{\label{tbl:long}}  
    
\section{公式定理示例}
	{\bfseries 行内公式示例。}工程技术上,把频率从\inmath{0}到\inmath{\omega_c}的范围定为这一低通滤波器的通频带(bandwidth,简写作~BW),而\inmath{\omega_c}则又称为截止频率(cut-off frequency)。在\inmath{\omega=\omega_c}时,相移为-\inmath{45\degree}。
	
	{\bfseries 单行公式示例。}引用\inmath{\omega_c=\frac{1}{\tau}}后,\inmath{H_u}可写为式\eqref{eqn:sline}:
    \begin{equation}\label{eqn:sline}
        H_u=\frac{1}{1+j\frac{\omega}{\omega_c}}
    \end{equation}
    
    {\bfseries 多组公式示例(包含矩阵等)。}矩阵行列式变换,如式\eqref{eqn:smatrix}。
    \begin{equation}
        \begin{split}
            & A = \begin{bmatrix} 4 & 6 \\ -6 & 8\end{bmatrix} \xrightarrow{R_{211}}
                  \begin{bmatrix} 4 & 6 \\ -2 & 14\end{bmatrix} \xrightarrow{R_{121}}
                  \begin{bmatrix} 4 & 20 \\ -2 & 14\end{bmatrix} \xrightarrow{R_{211}}
                  \begin{bmatrix} 2 & 20 \\ 0 & 34\end{bmatrix} \\
            &\phantom{A=} \xrightarrow{C_{21(-10)}}
                           \begin{bmatrix} 2 & 0 \\ 0 & 34\end{bmatrix} = D
        \end{split}
        \label{eqn:smatrix}
	\end{equation}
    
    {\bfseries 定理和证明示例。}“求真”是大学的内在本质。它是大学自身发展的内在规定性。“求真”有四层涵义,第一层涵义就是要拒绝假、丑、恶,以真理为友。五十年斗转星移,一代又一代成电人为了科学和真理的神圣事业坚毅执着、呕心泣血、不畏艰险、勇攀高峰,用心血和智慧浇灌出近千项具有国内领先或国际先进水平的科技成果。
    \begin{dingli}
        若\inmath{f(t)\leftrightarrow F(s)},则
        \begin{equation}
            f(t)e^{s_0 t} \leftrightarrow F(s-s_0)
        \end{equation}
        式中\inmath{s_0}为复常数。
    \end{dingli}
    \begin{zhengming}
        \[
            \int^\infty_0 f(t)e^{s_0 t}e^{-st} dt = \int^\infty_0 f(t)e^{-j(s-s_0)t} dt
            = F(s-s_0).
        \]
    \end{zhengming}
    
\section{枚举列表示例}
	{\bfseries 枚举环境示例。}可以说,成电五十年发展的历史,写就了成电人脚踏实地、百折不挠地追求、捍卫、弘扬真理,昌明和繁荣学术的光辉篇章。“求真”的第二层涵义是在探索科学规律的过程中要坚持科学的探索精神,科学的探索精神就是实事求是,遵循事物发展的内在规律,按规律办事,科学规划,科学决策,科学发展。
	
	\begin{enum}
    	\item 内容。
		\item 内容。“求真”的第三层涵义是要在工作中真抓真干,不走过场,不走形式,要围绕学校的中心工作,紧紧把握发展的主线,扎实工作,奋力推进各项事业。
    	\item 内容。
    \end{enum}
    
    {\bfseries 列表环境示例。}“求真”的第四层涵义是要求我们在人才培养上必须让学生具有真才实学,既掌握扎实的专业基础知识,又要培养学生的实践能力、创新能力,把学生真正造就成社会的栋梁。
    \begin{uitem}
      \item 内容;
      \item 内容;
      \begin{uitem}
        \item 第二层内容;
        \item 第二层内容;
      \end{uitem}
      \item 内容;
    \end{uitem}
    
    {\bfseries 术语描述列表示例。}“大气”是大学的内在精神。它是一所大学在长期的办学实践中所形成的精神特质,包括办学的理想与抱负、理念与传统、气魄与胸怀。“大气”就是要有崇尚科学追求真理的气质,要有奋发进取天下为先的气魄,要有海纳百川和谐包容的气度。
    \begin{desc}
        \item[New York] 纽约
        \item[Peking] 北京[Beijing的旧译]
        \item[脊髓外形与被膜] External features and meninges of the spinal cord(A,B,C)
        \item[注意] 用于专业术语的描述说明,说明内容不限于一行。
    \end{desc}

\section{算法与代码示例}
	插入较短的代码、算法描述等。@TODO

\section{引用示例}
	通常,在学术论文中,需要引述前人对相关工作已有的贡献,或者相关的数据、理论等。参考文献中只列作者直接阅读过、在论文中被引用过、正式发表的文献资料。参考文献按文中引用标注的顺序放在致谢后,不得放在各章之后。
	
	{\bfseries 文献直接引用示例。}如:爱因斯坦在文献\parencite{einstein1935}中提出的相对论,极大地改变了人类对宇宙和自然的“常识性”观念。霍金在其相关著作\cite{hawking1973}中对这一理论作了深入讨论。
	
	有时需要同时引用多篇文献,表示多位学者或者某一团队对某一问题所做的一系列研究。
	
	{\bfseries 多篇文献引用示例。}爱因斯坦提出相对论之后至今,有无数学者对其研究作了引证\cite{breitenlohner1984, brandt1997, misner1957, regge1957}。
	
	本文使用~biblatex-uestc~作为参考文献模板。该模板支持常用的文献类型,包括普通图书\cite{hawking1973}、论文集(会议文章)\cite{abramson1970}、科技报告\cite{wwwdevreport}、学位论文\cite{lichg2004}、期刊文献\cite{einstein1935}、电子文献\cite{scigen}、专利\cite{guoxb2012}。

	交叉引用。章、节、图、表等。@TODO

\section{本章小结}
	本章对模板中图、表、公式、定理、枚举、代码、引用环境的使用作了说明,并给出了相关示例。希望上述示例对模板使用者有一些帮助。
	
	本章中与模板无关的文字引自学校新闻中心的文章《求实求真大气大为—对校训的再思考》\cite{xiaoxun}。以下是该文的其余内容。
	
	\CJKfamily{kai} 就是要站得高,看得远,胸怀全局,从大处着眼,不求一时一事的得失,而求长远的成功。大学以传承知识、探索真理、昌明学术、启迪思想为目的,同时也具有引导社会价值,规范社会行为的使命,因此崇尚科学追求真理是大学自身使命的本质要求。其次,真理的探索既有成功的喜悦,也包含失败的痛苦,而一次成功的真理探索,往往孕育在无数的失败之中,这就要求无论作为教师还是学生,必须要有吃螃蟹的精神,要有敢为天下先的精神,要有不怕失败和挫折的勇气。再次,大学是群英荟萃之地,思想交汇之所,正如蔡元培先生所说:“大学者,‘囊括大典,网罗众家’之学府也。”在这里百家争鸣,百花齐放,竞相思辩,熠熠生辉,所以必须保持一种谦和平等的和相互学习的心态,兼收并蓄,兼容并包,所谓“江海不择细流,泰山不拒抔土”正是此理。而这种和谐包容的气度对当前构建和谐社会和构建和谐校园同样具有非常重要的现实意义。

	“大为”是大学的内在使命。它要求大学要对国家发展和社会进步作出大贡献,包括培养高层次的优秀人才,创造高水平的科技成果,提供高质量的社会服务。因此“大为”必须要有志存高远兼济天下的使命感,要有拒绝平庸追求卓越的精神境界,要有开拓创新成就事业的大作为。它从责任和使命,道德与品质,意志与精神等不同的角度为每一位成电师生给出了修炼的标尺。就个体而言,“大为”要求每一位师生要把自身的发展融入学校、融入到国家的发展之中,要在服务社会报效祖国的实践中实现人生的价值。就学校而言,成电五十年发展的历史,就是一部成电人矢志不渝地追求卓越、建设一流大学的奋进史,是一幅成电人爱国奉献、服务国防和国民经济建设的壮丽画卷,从~1956~年建校伊始,第一任院长和党委书记吴立人在首届开学典礼上描绘的把成都电讯工程学院办成我国乃至整个亚洲第一流的无线电大学的宏伟蓝图起,建设一流大学的办学理念就根植于成电人的心中,成为成电人不懈的精神追求和不断前进的强大精神动力。五十载滋兰树蕙,学校为社会培养了~8~万余名电子信息精英,为信息经济的发展做出了应有的贡献,一些学子或成为IT领军人物,或成为国防科技中坚,或成为企业家。近年来,学校更是在拒绝平庸追求卓越的精神倡导下,践行“大楼、大师、大气、大为”的办学理念,广大师生的精神面貌焕然一新,以改革的动力和创新的魄力聚精会神搞建设,一心一意谋发展,学校呈现出快速持续发展的良好态势,先后进入国家“211~工程”和“985~工程”重点建设大学行列,学科建设、师资队伍建设、人才培养、科学研究、社会服务等方面不断取得新突破,向着建设高水平研究型大学的目标昂首挺进。成电人用具体的行动诠释着“大为”的内涵。

	总之,“求实求真,大气大为”既是对传统的成电精神的传承,又是新的时代背景下对成电精神的弘扬;既是对学校教职工的劝导,又是对学生乃至校友的立身处世箴言;既是学校工作的标准和追求,又是个人品格修炼的基本标尺。另外从语言文字和表述上看,文字简洁贴切而寓意深刻,对仗工整而无雕琢痕迹,朴实厚重而大气磅礴,别开生面却纯正意远。可以说,“求实求真,大气大为”是成电精神和办学理念的高度概括,是成电人精神象征的高度抽象。

	“众里寻她千百度,蓦然回首,那人却在,灯火阑珊处。”或许,这就是我们为之企盼的“她”?

	一孔之见,在此仅作抛砖引玉,如有不妥之处,恳蒙惠。\CJKfamily{song}