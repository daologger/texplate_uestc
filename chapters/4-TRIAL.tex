%% 数值实验、实施过程

\chapter{数值实验}
	实验内容。
	
	本论文所做相关实验的相关内容阐述。包括实验装置和测试方法、经过整理加工的实验结果的分析讨论、与理论计算结果的比较,本研究方法与已有研究方法的比较等。
	
	论文应层次分明、数据可靠、文字简练、说明透彻、推理严谨、立论正确,避免使用文学性质的带感情色彩的非学术性词语。论文中如出现非通用性的新名词、新术语、新概念,应作相应解释。
	
	要严格执行~GB3100-3102-93~有关量和单位的规定(具体要求请参阅《常用量和单位》.计量出版社,1996)。论文中某一量的名称和符号应统一,量的符号、常量和变量符号必须采用斜体;计量单位名称的书写,可以采用国际通用符号,也可以用中文名称,但全文应统一,不要两种混用。计量单位符号,除用人名命名的单位第一个字母大写之外,一律用小写字母。
不定数字之后可用中文计量单位符号,如“几千克”。表达时刻时应采用中文计量单位,如“上午~9~点~1~刻”。计量单位符号一律采用正体书写。

\section{关于参考文献}
	文献搜索和所需条目。@TODO

\section{常见问题}
	\begin{enum}
		\item 段落之间出现多余空白(间距)
		\item 文字、URL、公式等超出页边距
		\item 图片错位
		\item 中英文间距不合理
	\end{enum}
	
	@TODO
	
\section{本章小结}
	本章对模板的使用作了补充说明,并针对常见的排版问题逐一给出了解决方法。