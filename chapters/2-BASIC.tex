%% 课题背景介绍

\chapter{基础介绍}
	基础介绍。
	
	论文主体是论文的主要部分,论文各章之间应该前后关联,构成一个有机的整体。论文给出的数据必须真实可靠,推理正确,结论明确,无概念性和科学错误。引用他人研究成果时,应注明出处,不得将其与本人的工作混淆在一起。

\section{一级标题}
	在各级标题之间,不宜直接用图片或公式定理等作为开篇。多数情况下需要对即将介绍的内容作基本介绍和引入。

\subsection{二级标题}
    论文主体内容。写作内容可因研究课题性质而不同,一般可包括:理论分析、计算方法、实验装置和测试方法、经过整理加工的实验结果的分析讨论、与理论计算结果的比较,本研究方法与已有研究方法的比较等。

\subsubsection{三级标题}
    正文内容。
    
    通常在绪论之后,对研究内容的背景作基本介绍,以便于引入本文作者的实际研究课题。
    
\section{本章小结}
    论文主体各章后应有一节“本章小结”,是对各章研究内容、方法与成果的简洁准确的总结与概括,也是学位论文最后结论的依据。