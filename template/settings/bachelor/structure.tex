\AtBeginDocument{ % 插入到论文内容的最前面。
    \begin{CJK}{UTF8}{rm} % 中文支持环境,此处内容不受ctexbook文类影响,所以必须手动加这个环境
%    %% 封面页
\title{电子科技大学硕士毕业论文~\LaTeX{}~模板说明} % 论文题目
\author{作者姓名} % 作者姓名
\stuid{201XXXXXXXXX} % 学号
\major{专业名称} % 专业
\school{所属学院} % 学院
\adviser{指导教师姓名}{指导教师职称}{电子科技大学} % 指导教师
\university{电子科技大学} % 学校
\date{2013}{3}{XX} % 日期(年月日)

\englishtitle{ENGLISH TITLE OF MASTER THESIS} % 论文的英文名
\classnumber{K825.1-64} % 分类号
\securityclassification{公开} % 密级
\UDC{676.874} %《国际十进分类法UDC》的类号
\adviserB{某老师}{教授}{电子科技大学} % 第二指导教师
\adviserC{某老师}{教授}{电子科技大学} % 第二指导教师
\adviserD{某老师}{教授}{电子科技大学} % 第二指导教师
\oraldefensedate{2013}{5}{XX} % 论文答辩日期
\awarddate{2013}{6}{XX} % 学位授予日期
\chairman{某老师} % 答辩委员会主席
\appraiser{某老师、某老师、某老师} % 评阅人
\englishmajor{Major} % 专业英文名
\englishauthor{AuthorName} % 作者英文名
\englishadvisor{Advisor Name} % 指导教师英文名
\englishshcool{School/Institution} % 学院英文名
 % 导入封面信息
%    \maketitle % 生成并插入封面
    \Cabstractmatter % 设置中文摘要版式
    %% 中文摘要

\begin{abstractCN}{关键词一}{关键词二}{关键词三}{关键词四}{关键词五-后面的文字用于调整关键词换行效果}
	摘要内容。
	
	硕士论文中文摘要约~800~字左右,博士论文中文摘要约~1500~字左右。内容应包括工作目的、研究方法、成果和结论,要突出本论文的创造性成果,语言力求精炼。为了便于文献检索,应在本页下方另起一行注明论文的关键词(3-5~个)。
	
	摘要中不要列举例证,不讲研究过程,不用图表,不给化学结构式,也不要作自我评价。撰写论文摘要的常见毛病,一是照搬论文正文中的小标题(目录)或论文结论部分的文字;二是内容不浓缩、不概括,文字篇幅过长\cite{zhaiyao}。
\end{abstractCN} % 插入中文摘要
    \Eabstractmatter % 设置英文摘要版式
    %% 英文摘要

\begin{abstractEN}{keyword-1}{keyword-2}{keyword-3}{keyword-4}{keyword-5-and more keywords to go for the next line}
	Abstract content.
	
	Sample: In recent years, much research has been devoted to the development of model checking; contrarily, few have analyzed the investigation of A* search. Given the current status of atomic methodologies, analysts obviously desire the synthesis of congestion control, which embodies the typical principles of networking. In order to fulfill this mission, we disconfirm that even though the memory bus  and local-area networks can synchronize to accomplish this aim, the much-touted relational algorithm for the analysis of Moore's Law is in Co-NP. [This sample is generated via scigen\cite{scigen}.]
\end{abstractEN}
 % 插入英文摘要
    \tocmatter % 目录版式
    \setcounter{tocdepth}{2} % 设置目录深度
    \tableofcontents % 插入目录
    \ifblank{\ifusingglossary}{}{ % 若文中使用\fuhao命令,则插入主要符号表
        \glossarymatterfancy % 设置主要符号表版式
        \printglossary % 插入主要符号表
        \mainmatter % 修正主要符号表版式,这里是受目录版式的影响,必须在此加入\mainmatter
    }
    \end{CJK}
    \mainmatter % 正文区版式
}

\AtEndDocument{ % 插入到论文内容最后面。
    \begin{CJK}{UTF8}{rm}
    \phantomsection % 手动添加目录项之前需要这个命令,用以更新目录超链接的跳转页码。
    \chapter*{致\enspace 谢} % 用带星号的章命令插入不带章号的致谢
    \addcontentsline{toc}{chapter}{致\enspace 谢} % 将致谢编入目录
    \markboth{致\enspace 谢}{} % 页眉设置 
    %% 致谢内容
	
	致谢对象限于在学术方面对论文的完成有较重要帮助的团体和个人(不超过500字)。	
 % 插入致谢内容
    \clearpage
    \phantomsection % 手动添加目录项之前需要这个命令,用以更新目录超链接的跳转页码。
    \addcontentsline{toc}{chapter}{参考文献} % 将参考文献编入目录
    {\zihao{5}{
        \printbibliography[
            title={参考文献}
        ] % 插入参考文献
    }}
    \appendix % 附录版式
    \renewcommand{\chaptermark}[1]{
        \markboth{附录~\Alph{chapter}~\quad  #1 }{}
    } % 附录页眉设置
    %% 附录部分

	附录内容。
	
	论文中较长篇幅的程序源代码或实验结果图像通常作为附录部分,附在论文最后。可以包括正文内不便列出的冗长公式推导;以备他人阅读方便所需的辅助性数学工具或表格;重复性数据图表;计算程序及说明等。

\section*{init.s~源码}
	以下代码为第一版~Unix~系统的部分汇编源码。详见~\url{http://minnie.tuhs.org/cgi-bin/utree.pl?file=V1/init.s}
	\codes{}{init.s}


	
	
	 % 插入附录内容
    \IfFileExists{contents/original.tex}{
        \clearpage
        \renewcommand{\chaptermark}[1]{\markboth{外文资料原文}{}}
        \def\leftmark{外文资料原文}
        %% 英文原文

\chapter*{The Name of the Game}
English words like `technology' stem from a Greek root beginning with
the letters $\tau\epsilon\chi\ldots\,$; and this same Greek word means {\sl
art\/} as well as technology. Hence the name \TeX, which is an
uppercase form of $\tau\epsilon\chi$.{TeX (actually \TeX), meaning of}
$\tau\epsilon\chi$

Insiders pronounce the $\chi$ of \TeX\ as a Greek chi, not as an `x', so that
\TeX\ rhymes with the word blecchhh. It's the `ch' sound in Scottish words
like {\sl loch\/} or German words like {\sl ach\/}; it's a Spanish `j' and a
Russian `kh'. When you say it correctly to your computer, the terminal
may become slightly moist.

The purpose of this pronunciation exercise is to remind you that \TeX\ is
primarily concerned with high-quality technical manuscripts: Its emphasis is
on art and technology, as in the underlying Greek word. If you merely want
to produce a passably good document---something acceptable and basically
readable but not really beautiful---a simpler system will usually suffice.
With \TeX\ the goal is to produce the {\sl finest\/} quality; this requires
more attention to detail, but you will not find it much harder to go the
extra distance, and you'll be able to take special pride in the finished
product. 

On the other hand, it's important to notice another thing about \TeX's name:
The `E' is out of kilter. This {logo}
displaced `E' is a reminder that \TeX\ is about typesetting, and it
distinguishes \TeX\ from other system names. In fact, {TEX} (pronounced
{\sl tecks\/}) is the admirable {\sl Text EXecutive\/} processor developed by
{Honeywell Information Systems}. Since these two system names are
{Bemer, Robert, see TEX, ASCII}
pronounced quite differently, they should also be spelled differently. The
correct way to refer to \TeX\ in a computer file, or when using some other
medium that doesn't allow lowering of the `E', is to type `|TeX|'. Then
there will be no confusion with similar names, and people will be
primed to pronounce everything properly.
        \renewcommand{\chaptermark}[1]{\markboth{外文资料译文}{}}
        \def\leftmark{外文资料译文}
        %% 英文译文

\chapter*{此名有诗意}
英语单词“technology”来源于以字母$\tau\epsilon\chi\ldots\,$开头的希腊词根;并且这个希腊单词除了
technology的意思外也有art的意思。因此,名称TEX是$\tau\epsilon\chi$的大写格式。

在发音时,\TeX 的$\chi$的发音与希腊的chi一样,而不是“x”,所以\TeX 与blecchhh 押韵。“ch”
听起来象苏格兰单词中的loch 或者德语单词中的ach;它在西班牙语中是“j”,在俄语中是“kh”。
当你对着计算机正确读出时, 终端屏幕上可能有点雾。

这个发音练习是提醒你,\TeX 主要处理的是高质量的专业书稿:它的重点在艺术和专业方
面, 就象希腊单词的含义一样。如果你仅仅想得到一个过得去——可读下去但不那么漂亮——的
文书, 那么简单的系统一般就够用了。使用\TeX 的目的是得到最好的质量;这就要在细节上花功
夫, 但是你不会认为它难到哪里去,并且你会为所完成的作品感到特别骄傲。

另一方面重要的是要注意到与\TeX 名称有关的另一件事: “E”是错位的。这个偏移“E”的标
识提醒人们,\TeX 与排版有关,并且把\TeX 从其它系统的名称区别开来。实际上,TEX(读音为
tecks)是Honeywell Information Systems 的极好的Text EXecutive处理器。因为这两个系统的
名称读音差别很大,所以它们的拼写也不同。在计算机中表明\TeX 文件的正确方法,或者当所用
的方式无法降低“E”时,就要写作“TeX”。这样, 就与类似的名称不会产生混淆, 并且为人们可以
正确发音提供了条件。
    }{}
    \clearpage
    \end{CJK}
}