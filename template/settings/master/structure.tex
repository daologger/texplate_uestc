\AtBeginDocument{ % 插入到论文内容的最前面。
    \begin{CJK}{UTF8}{rm} % 中文支持环境,此处内容不受ctexbook文类影响,所以必须手动加这个环境
    %% 封面页
\title{电子科技大学硕士毕业论文~\LaTeX{}~模板说明} % 论文题目
\author{作者姓名} % 作者姓名
\stuid{201XXXXXXXXX} % 学号
\major{专业名称} % 专业
\school{所属学院} % 学院
\adviser{指导教师姓名}{指导教师职称}{电子科技大学} % 指导教师
\university{电子科技大学} % 学校
\date{2013}{3}{XX} % 日期(年月日)

\englishtitle{ENGLISH TITLE OF MASTER THESIS} % 论文的英文名
\classnumber{K825.1-64} % 分类号
\securityclassification{公开} % 密级
\UDC{676.874} %《国际十进分类法UDC》的类号
\adviserB{某老师}{教授}{电子科技大学} % 第二指导教师
\adviserC{某老师}{教授}{电子科技大学} % 第二指导教师
\adviserD{某老师}{教授}{电子科技大学} % 第二指导教师
\oraldefensedate{2013}{5}{XX} % 论文答辩日期
\awarddate{2013}{6}{XX} % 学位授予日期
\chairman{某老师} % 答辩委员会主席
\appraiser{某老师、某老师、某老师} % 评阅人
\englishmajor{Major} % 专业英文名
\englishauthor{AuthorName} % 作者英文名
\englishadvisor{Advisor Name} % 指导教师英文名
\englishshcool{School/Institution} % 学院英文名
 % 导入封面信息
    \maketitle % 生成并插入封面
    \Cabstractmatter % 设置中文摘要版式
    %% 中文摘要

\begin{abstractCN}{关键词一}{关键词二}{关键词三}{关键词四}{关键词五-后面的文字用于调整关键词换行效果}
	摘要内容。
	
	硕士论文中文摘要约~800~字左右,博士论文中文摘要约~1500~字左右。内容应包括工作目的、研究方法、成果和结论,要突出本论文的创造性成果,语言力求精炼。为了便于文献检索,应在本页下方另起一行注明论文的关键词(3-5~个)。
	
	摘要中不要列举例证,不讲研究过程,不用图表,不给化学结构式,也不要作自我评价。撰写论文摘要的常见毛病,一是照搬论文正文中的小标题(目录)或论文结论部分的文字;二是内容不浓缩、不概括,文字篇幅过长\cite{zhaiyao}。
\end{abstractCN} % 插入中文摘要
    \Eabstractmatter % 设置英文摘要版式
    %% 英文摘要

\begin{abstractEN}{keyword-1}{keyword-2}{keyword-3}{keyword-4}{keyword-5-and more keywords to go for the next line}
	Abstract content.
	
	Sample: In recent years, much research has been devoted to the development of model checking; contrarily, few have analyzed the investigation of A* search. Given the current status of atomic methodologies, analysts obviously desire the synthesis of congestion control, which embodies the typical principles of networking. In order to fulfill this mission, we disconfirm that even though the memory bus  and local-area networks can synchronize to accomplish this aim, the much-touted relational algorithm for the analysis of Moore's Law is in Co-NP. [This sample is generated via scigen\cite{scigen}.]
\end{abstractEN}
 % 插入英文摘要
    \tocmatter % 目录版式
    \setcounter{tocdepth}{3} % 设置目录深度
    \tableofcontents % 插入目录
    \clearpage
    \ifblank{\ifusingglossary}{}{ % 若文中使用\fuhao命令,则插入主要符号表
        \glossarymatterfancy % 设置主要符号表版式
        \printglossary % 插入主要符号表
        \mainmatter % 修正主要符号表版式,这里是受目录版式的影响,必须在此加入\mainmatter
    }
    \end{CJK}
    \mainmatter % 正文区版式
    %% 令图引用(如图X-X)中'X-X'与中文部分有一定间隙
	\let\oldref\ref
	\makeatletter
	\newcommand\myref[1]{\,\oldref{#1}\,}
	\newcommand\myeqref[1]{\!(\!\oldref{#1}\!)}
	\makeatother
	\renewcommand{\ref}[1]{\myref{#1}}
	\renewcommand{\eqref}[1]{\myeqref{#1}}
}

\AtEndDocument{ % 插入到论文内容最后面。
    \begin{CJK}{UTF8}{rm}
    \backmatter
    \chapter{致\enspace 谢} % 用带星号的章命令插入不带章号的致谢
    \markboth{致谢}{} % 致谢页眉设置 
    %% 致谢内容
	
	致谢对象限于在学术方面对论文的完成有较重要帮助的团体和个人(不超过500字)。	
 % 插入致谢内容
    \newpage
    \phantomsection % 手动添加目录项之前需要这个命令,用以更新目录超链接的跳转页码。
    \addcontentsline{toc}{chapter}{参考文献} % 将参考文献编入目录
    {\sloppy\zihao{5}{ % 
        \printbibliography[ %
            title={参考文献} %
         ] % 插入参考文献
    }} % 参考文献字号为小四
	\IfFileExists{contents/appendix.tex}{ %
		\chapter{附\enspace 录} %
    	\markboth{附录}{} % 附录页眉设置 
    	%% 附录部分

	附录内容。
	
	论文中较长篇幅的程序源代码或实验结果图像通常作为附录部分,附在论文最后。可以包括正文内不便列出的冗长公式推导;以备他人阅读方便所需的辅助性数学工具或表格;重复性数据图表;计算程序及说明等。

\section*{init.s~源码}
	以下代码为第一版~Unix~系统的部分汇编源码。详见~\url{http://minnie.tuhs.org/cgi-bin/utree.pl?file=V1/init.s}
	\codes{}{init.s}


	
	
	 % 插入附录内容	
	}{}
    \IfFileExists{bib/pub.bib}{ % 如果有pub.bib,将显示这一章
        \clearpage
        \phantomsection % 手动添加目录项之前需要这个命令,用以更新目录超链接的跳转页码。
        \addcontentsline{toc}{chapter}{\pubname@degree} % 将个人著作编入目录
        \newrefsection[bib/pub.bib]
        \nocite{*}
        {\sloppy\zihao{5}{
            \printbibliography[
                title={\pubname@degree}
            ] % 插入个人著作
        }} % 参考文献字号为小四
        \endrefsection
    }{}
    \IfFileExists{chapters/X-FRONT.tex}{%
		\chapter{写在最后} %
    	\markboth{}{} %
    	%% 模板版本介绍

	写在最后的话。
	
	模板的由来,制作过程。需要的基础知识,所能带来的好处。版本历史。 %
    }{} % 
    \clearpage %
    \end{CJK}
}